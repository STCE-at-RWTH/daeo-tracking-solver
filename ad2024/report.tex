%%%%%%%%%%%%%%%%%%%%%%%%%%  ltexpprt_twocolumn.tex  %%%%%%%%%%%%%%%%%%%%%%%%%%%%%%%%
%
% This is ltexpprt_twocolumn.tex, an example file for use with the SIAM LaTeX2E
% Preprint Series macros. It is designed to provide two-column output.
% Please take the time to read the following comments, as they document
% how to use these macros. This file can be composed and printed out for
% use as sample output.

% Any comments or questions regarding these macros should be directed to:
%
%                 Rachel Ginder
%                 SIAM
%                 3600 University City Science Center
%                 Philadelphia, PA 19104-2688
%                 USA
%                 Telephone: (215) 382-9800
%                 Fax: (215) 386-7999
%                 e-mail: rginder@siam.org


% This file is to be used as an example for style only. It should not be read
% for content.

%%%%%%%%%%%%%%% PLEASE NOTE THE FOLLOWING STYLE RESTRICTIONS %%%%%%%%%%%%%%%

%%  1. There are no new tags.  Existing LaTeX tags have been formatted to match
%%     the Preprint series style.
%%
%%  2. Do not change the margins or page size!  Do not change from the default
%%     text font!
%%
%%  3. You must use \cite in the text to mark your reference citations and
%%     \bibitem in the listing of references at the end of your chapter. See
%%     the examples in the following file. If you are using BibTeX, please
%%     supply the bst file with the manuscript file.
%%
%%  4. This macro is set up for two levels of headings (\section and
%%     \subsection). The macro will automatically number the headings for you.
%%
%%  5. No running heads are to be used for this volume.
%%
%%  6. Theorems, Lemmas, Definitions, Equations, etc. are to be double numbered,
%%     indicating the section and the occurrence of that element
%%     within that section. (For example, the first theorem in the second
%%     section would be numbered 2.1. The macro will
%%     automatically do the numbering for you.
%%
%%  7. Figures and Tables must be single-numbered.
%%     Use existing LaTeX tags for these elements.
%%     Numbering will be done automatically.
%%
%%  8. Page numbering is no longer included in this macro.
%%     Pagination will be set by the program committee.
%%
%%
%%%%%%%%%%%%%%%%%%%%%%%%%%%%%%%%%%%%%%%%%%%%%%%%%%%%%%%%%%%%%%%%%%%%%%%%%%%%%%%



\documentclass[twoside,leqno,twocolumn]{article}

% Comment out the line below if using A4 paper size
\usepackage[letterpaper]{geometry}
\usepackage{amssymb}
\usepackage{amsmath}
\usepackage{accents}
\usepackage{biblatex}
\addbibresource{daeos.bib}
\usepackage{ltexpprt}
\usepackage{hyperref}

\DeclareMathOperator*{\argmin}{\arg\min}
\newcommand{\bbR}{\ensuremath{\mathbb{R}}}
\newcommand\ubar[1]{\underaccent{\bar}{#1}}

%% add to the AMS macros
\newtheorem{assumption}{Assumption}[section]

\begin{document}

%
\newcommand\relatedversion{}
% \renewcommand\relatedversion{\thanks{The full version of the paper can be accessed at \protect\url{https://arxiv.org/abs/1902.09310}}} % Replace URL with link to full paper or comment out this line


%\setcounter{chapter}{2} % If you are doing your chapter as chapter one,
%\setcounter{section}{3} % comment these two lines out.

\title{\Large An Efficient Local Optimizer-Tracking Solver for Differential-Algebriac Equations with Optimization Criteria\relatedversion}
% which affiliation should I put here?
\author{Alexander Fleming\thanks{RWTH Aachen University.}
\and Jens Deussen\thanks{RWTH Aachen University.}}

\date{}

\maketitle

% Copyright Statement
% When submitting your final paper to a SIAM proceedings, it is requested that you include
% the appropriate copyright in the footer of the paper.  The copyright added should be
% consistent with the copyright selected on the copyright form submitted with the paper.
% Please note that "20XX" should be changed to the year of the meeting.

% Default Copyright Statement
% \fancyfoot[R]{\scriptsize{Copyright \textcopyright\ 20XX by SIAM\\
% Unauthorized reproduction of this article is prohibited}}

% Depending on which copyright you agree to when you sign the copyright form, the copyright
% can be changed to one of the following after commenting out the default copyright statement
% above.

\fancyfoot[R]{\scriptsize{Copyright \textcopyright\ 2024\\
Copyright for this paper is retained by authors}}

%\fancyfoot[R]{\scriptsize{Copyright \textcopyright\ 20XX\\
%Copyright retained by principal author's organization}}

%\pagenumbering{arabic}
%\setcounter{page}{1}%Leave this line commented out.

% send help
\begin{abstract} \small\baselineskip=9pt A solver for differential-algebraic equations with embedded optimization criteria (DAEOs) was developed. The new solver relies on the reduction of a DAEO to a sequence of differential inclusions separated by "jump events". Between jump events, the solver relies solely on an integrator, and it explicitly treats jump events when they occur. This preserves the order of convergence of the integrator component without sacrificing performance to perform global optimization at (nearly) every time step.
\end{abstract}


\section{Problem Setting.}
A differential-algebraic equations with embedded optimization criteria (DAEO) is written as an initial-value problem like so:
\begin{equation} \label{eq:daeo-ivp}
\begin{aligned}
	x(t_0) &= x_0\\
	\dot{x}(t, x, y) &= f(x(t), y(t))\\
	\left\{y_k\right\}(t, x) &= \argmin_{y}h(x(t), y)
\end{aligned}
\end{equation}
where $f:\bbR^{n_x}\times\bbR^{n_y}$ describes the differential behavior of the system, and $h:\bbR^{n_x}\times\bbR^{n_y}$ is an objective function to be minimized. At any given time $t>t_0$, the set of $k$ minimizers of $h$ is denoted by $\left\{y_k\right\}$. The notation $\dot{x}$ denotes $\partial x/\partial t$, $\partial_x f$ denotes $\partial f/\partial x$, and $d_x f$ denotes $df/dx$, in the cases where the difference between the partial and total derivatives is relevant. Second derivatives are written similarly, with $\ddot{x}$ for $\partial^2_t x$ and $\partial^2_{yx} f$ for $\partial f/\partial x\partial y$.

At any time $t > t_0$, there will be a some $y^\star(t)\in\left\{y_k(t)\right\}$ such that
\begin{equation}
	h(x(t), y^\star(t))\leq h(x(t),y_i(t))\, \forall\,y_i(t)\in\left\{y_k(t)\right\} 
\end{equation}
% TODO get someone to check the terminology here!
This implies that the initial value problem posed in \eqref{eq:daeo-ivp} can be divided into a sequence of initial-value problems if $y^\star(t)$ has a sufficiently well-behaved dependency on $x$ and $t$.

\section{Solution Technique.}
To solve this sequence of IVPs generated by the jump events, the solver must 
\begin{enumerate}
	\item Generate the set $\left\{y_k(t)\right\}$ at $t=t_0$ and at user-specified times $t>t_0$ to correct for any integration error that may accumulate.
	\item Detect when the identity of $y^\star$ changes and correct the time step in which it happens.
	\item Solve the sequence of IVPs after $t_0$ without using the global optimizer.
\end{enumerate}

\subsection{Global Optimization with Interval Arithmetic.}
\begin{assumption} 
	\label{assume:twice-lipschitz}
	Both $f$ and $h$ are twice Lipschitz continuous with respect to $(x(t), y(t))$.
\end{assumption}
Optimality conditions for $h(x(t), y)$ follow from \ref{assume:twice-lipschitz}
\begin{equation}
	\label{eq:optimality-conditions}
	\begin{aligned}
		0 &= \partial_yh(x, y)\\
		0 &\prec\partial^2_{y}h(x, y)
	\end{aligned}
\end{equation}
The global optimizer must find all points $y_i$ that satisfy these optimality conditions. Using a suitable algorithmic differentiation (AD) tool and an interval arithmetic library allows for the comparatively inexpensive verification of the optimality conditions. 
\begin{Definition}
	An interval evaluation of a function $\left[\ubar{y},\bar{y}\right] = f(\left[\ubar x, \bar x\right])$ must yield an interval that contains all pointwise evaluations $f(x)\,\forall\,x\in\left[\ubar x, \bar x\right]$. \cite{hickeyIntervalArithmeticPrinciples2001}
\end{Definition}






\subsection{Jump Event Detection}

\subsection{Correcting the Integrator}

\section{Design Considerations.}Several good ordering algorithms (nested dissection and
minimum degree)
are available for computing $P$  \cite{GEORGELIU}, \cite{ROSE72}.
Since our interest here does not
focus directly on the ordering, we assume for convenience that $P=I$,
or that $A$ has been preordered to reflect an appropriate choice of $P$.

Our purpose here is to examine the nonnumerical complexity of the
sparse elimination algorithm given in  \cite{BANKSMITH}.
As was shown there, a general sparse elimination scheme based on the
bordering algorithm requires less storage for pointers and
row/column indices than more traditional implementations of general
sparse elimination.  This is accomplished by exploiting the m-tree,
a particular spanning tree for the graph of the filled-in matrix.

\begin{theorem} The method  was extended to three
dimensions. For the standard multigrid
coarsening
(in which, for a given grid, the next coarser grid has $1/8$
as many points), anisotropic problems require plane
relaxation to
obtain a good smoothing factor.\end{theorem}

Our purpose here is to examine the nonnumerical complexity of the
sparse elimination algorithm given in  \cite{BANKSMITH}.
As was shown there, a general sparse elimination scheme based on the
bordering algorithm requires less storage for pointers and
row/column indices than more traditional implementations of general
sparse elimination.  This is accomplished by exploiting the m-tree,
a particular spanning tree for the graph of the filled-in matrix.
Several good ordering algorithms (nested dissection and minimum degree)
are available for computing $P$  \cite{GEORGELIU}, \cite{ROSE72}.
Since our interest here does not
focus directly on the ordering, we assume for convenience that $P=I$,
or that $A$ has been preordered to reflect an appropriate choice of $P$.

\begin{proof} In this paper we consider two methods. The first method
is
basically the method considered with two differences:
first, we perform plane relaxation by a two-dimensional
multigrid method, and second, we use a slightly different
choice of
interpolation operator, which improves performance
for nearly singular problems. In the second method coarsening
is done by successively coarsening in each of the three
independent variables and then ignoring the intermediate
grids; this artifice simplifies coding considerably.
\end{proof}

Our purpose here is to examine the nonnumerical complexity of the
sparse elimination algorithm given in  \cite{BANKSMITH}.
As was shown there, a general sparse elimination scheme based on the
bordering algorithm requires less storage for pointers and
row/column indices than more traditional implementations of general
sparse elimination.  This is accomplished by exploiting the m-tree,
a particular spanning tree for the graph of the filled-in matrix.

\begin{Definition}{\rm We describe the two methods in \S 1.2. In \S\ 1.3. we
discuss
some remaining details.}
\end{Definition}

Our purpose here is to examine the nonnumerical complexity of the
sparse elimination algorithm given in  \cite{BANKSMITH}.
As was shown there, a general sparse elimination scheme based on the
bordering algorithm requires less storage for pointers and
row/column indices than more traditional implementations of general
sparse elimination.  This is accomplished by exploiting the m-tree,
a particular spanning tree for the graph of the filled-in matrix.
Several good ordering algorithms (nested dissection and minimum degree)
are available for computing $P$  \cite{GEORGELIU}, \cite{ROSE72}.
Since our interest here does not
focus directly on the ordering, we assume for convenience that $P=I$,
or that $A$ has been preordered to reflect an appropriate choice of $P$.

Our purpose here is to examine the nonnumerical complexity of the
sparse elimination algorithm given in  \cite{BANKSMITH}.
As was shown there, a general sparse elimination scheme based on the
bordering algorithm requires less storage for pointers and
row/column indices than more traditional implementations of general
sparse elimination.

\begin{lemma} We discuss first the choice for $I_{k-1}^k$
which is a generalization. We assume that $G^{k-1}$ is
obtained
from $G^k$
by standard coarsening; that is, if $G^k$ is a tensor product
grid $G_{x}^k \times G_{y}^k \times G_{z}^k$,
$G^{k-1}=G_{x}^{k-1} \times G_{y}^{k-1} \times G_{z}^{k-1}$,
where $G_{x}^{k-1}$ is obtained by deleting every other grid
point of $G_x^k$ and similarly for $G_{y}^k$ and $G_{z}^k$.
\end{lemma}

To our knowledge, the m-tree previously has not been applied in this
fashion to the numerical factorization, but it has been used,
directly or indirectly, in several optimal order algorithms for
computing the fill-in during the symbolic factorization phase
[4] - [10], [5], [6]. In \S 1.3., we analyze the complexity of the old and new
approaches to the intersection problem for the special case of
an $n \times n$ grid ordered by nested dissection. The special
structure of this problem allows us to make exact estimates of
the complexity. To our knowledge, the m-tree previously has not been applied in this
fashion to the numerical factorization, but it has been used,
directly or indirectly, in several optimal order algorithms for
computing the fill-in during the symbolic factorization phase
[4] - [10], [5], [6].

In \S 1.2, we review the bordering algorithm, and introduce
the sorting and intersection problems that arise in the
sparse formulation of the algorithm.
In \S 1.3., we analyze the complexity of the old and new
approaches to the intersection problem for the special case of
an $n \times n$ grid ordered by nested dissection. The special
structure of this problem allows us to make exact estimates of
the complexity. To our knowledge, the m-tree previously has not been applied in this
fashion to the numerical factorization, but it has been used,
directly or indirectly, in several optimal order algorithms for
computing the fill-in during the symbolic factorization phase
[4] - [10], [5], [6].


For the old approach, we show that the
complexity of the intersection problem is $O(n^{3})$, the same
as the complexity of the numerical computations.  For the
new approach, the complexity of the second part is reduced to
$O(n^{2} (\log n)^{2})$.

To our knowledge, the m-tree previously has not been applied in this
fashion to the numerical factorization, but it has been used,
directly or indirectly, in several optimal order algorithms for
computing the fill-in during the symbolic factorization phase
[4] - [10], [5], [6]. In \S 1.3., we analyze the complexity of the old and new
approaches to the intersection problem for the special case of
an $n \times n$ grid ordered by nested dissection. The special
structure of this problem allows us to make exact estimates of
the complexity. To our knowledge, the m-tree previously has not been applied in this
fashion to the numerical factorization, but it has been used,
directly or indirectly, in several optimal order algorithms for
computing the fill-in during the symbolic factorization phase
[4] - [10], [5], [6].
This is accomplished by exploiting the m-tree,
a particular spanning tree for the graph of the filled-in matrix.
To our knowledge, the m-tree previously has not been applied in this
fashion to the numerical factorization, but it has been used,
directly or indirectly, in several optimal order algorithms for
computing the fill-in during the symbolic factorization phase
\cite{EISENSTAT} - \cite{LIU2}, \cite{ROSE76},  \cite{SCHREIBER}.

\subsection{Robustness.}We do not
attempt to present an overview
here, but rather attempt to focus on those results that
are relevant to our particular algorithm.
This section assumes prior knowledge of the role of graph theory
in sparse Gaussian elimination; surveys of this role are
available in \cite{ROSE72} and \cite{GEORGELIU}. More general
discussions of elimination trees are given in
\cite{LAW} - \cite{LIU2}, \cite{SCHREIBER}.
Thus, at the $k$th stage, the bordering algorithm consists of
solving the lower triangular system
\begin{equation} \label{1.2}
 L_{k-1}v = c
\end{equation}
and setting
\begin{eqnarray}
\ell &=& D^{-1}_{k-1}v , \\
\delta &=& \alpha - \ell^{t} v .
\end{eqnarray}

\begin{figure}
\vspace{14pc}
\caption{This is a figure 1.1.}
\end{figure}

\section{Robustness.} We do not
attempt to present an overview
here, but rather attempt to focus on those results that
are relevant to our particular algorithm.

\subsection{Versatility.}The special
structure of this problem allows us to make exact estimates of
the complexity.  For the old approach, we show that the
complexity of the intersection problem is $O(n^{3})$, the same
as the complexity of the numerical computations
\cite{GEORGELIU}, \cite{ROSEWHITTEN}.  For the
new approach, the complexity of the second part is reduced to
$O(n^{2} (\log n)^{2})$.

To our knowledge, the m-tree previously has not been applied in this
fashion to the numerical factorization, but it has been used,
directly or indirectly, in several optimal order algorithms for
computing the fill-in during the symbolic factorization phase
[4] - [10], [5], [6]. In \S 1.3., we analyze the complexity of the old and new
approaches to the intersection problem for the special case of
an $n \times n$ grid ordered by nested dissection. The special
structure of this problem allows us to make exact estimates of
the complexity. To our knowledge, the m-tree previously has not been applied in this
fashion to the numerical factorization, but it has been used,
directly or indirectly, in several optimal order algorithms for
computing the fill-in during the symbolic factorization phase
[4] - [10], [5], [6].

In \S 1.2, we review the bordering algorithm, and introduce
the sorting and intersection problems that arise in the
sparse formulation of the algorithm.
In \S 1.3., we analyze the complexity of the old and new
approaches to the intersection problem for the special case of
an $n \times n$ grid ordered by nested dissection. The special
structure of this problem allows us to make exact estimates of
the complexity. To our knowledge, the m-tree previously has not been applied in this
fashion to the numerical factorization, but it has been used,
directly or indirectly, in several optimal order algorithms for
computing the fill-in during the symbolic factorization phase
[4] - [10], [5], [6].


For the old approach, we show that the
complexity of the intersection problem is $O(n^{3})$, the same
as the complexity of the numerical computations.  For the
new approach, the complexity of the second part is reduced to
$O(n^{2} (\log n)^{2})$.

To our knowledge, the m-tree previously has not been applied in this
fashion to the numerical factorization, but it has been used,
directly or indirectly, in several optimal order algorithms for
computing the fill-in during the symbolic factorization phase
[4] - [10], [5], [6]. In \S 1.3., we analyze the complexity of the old and new
approaches to the intersection problem for the special case of
an $n \times n$ grid ordered by nested dissection. The special
structure of this problem allows us to make exact estimates of
the complexity. To our knowledge, the m-tree previously has not been applied in this
fashion to the numerical factorization, but it has been used,
directly or indirectly, in several optimal order algorithms for
computing the fill-in during the symbolic factorization phase
[4] - [10], [5], [6].
This is accomplished by exploiting the m-tree,
a particular spanning tree for the graph of the filled-in matrix.
To our knowledge, the m-tree previously has not been applied in this
fashion to the numerical factorization, but it has been used,
directly or indirectly, in several optimal order algorithms for
computing the fill-in during the symbolic factorization phase
\cite{EISENSTAT} - \cite{LIU2}, \cite{ROSE76},  \cite{SCHREIBER}.

\printbibliography
\end{document}

% End of ltexpprt.tex 